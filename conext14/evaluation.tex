%!TEX root =conext14.tex
\section{Performance Evaluation}
\label{sec:evaluation}
To evaluate the performance of our proposed system, we implemented our multipath IP in Ubuntu under Linux kernel $3.12.1$. Two desktops are installed with the MPIP enabled Ubuntu system. At each desktop, two NICs working at $25$Mbits/sec are installed which means that there are totally $4$ paths and the throughput upper bound is $50$Mbits/sec between the two nodes. We modify an open source software \bf{Simple Traffic Monitor}\cite{simon01} to record the real time traffic that goes through each NIC.

%For mobile devices, we implement this feature in Google Nexus $4$ with Android $4.01$.

\subsection{TCP/UDP throughput enhancement}
\label{sec:tcp}

In this section, we try to verify that our system can achieve high throughput in both TCP and UDP scenarios. As a typical implementation of multipath, MPTCP creates the highest TCP throughput record between two nodes in \cite{record}. In that demonstration, they used $6$ $1$Gbps NIC interfaces at each node, and connect them directly without only middle boxes, and they limited the number of paths to $6$ by setting up IP TABLES in Ubuntu. Also, they did bunch of TCP parameter optimization to squeeze out all possible throughout. They achieved a breathtaking $51$Gbps throughput in that demonstration. 

We don't have the same configuration of the record-breaking plat of MPTCP, but we use the same typical configuration for all scenarios to do side-by-side configuration. We try to verify that our implementation can achieve the same throughput improvement as MPTCP for TCP traffic, and also our system can have the same enhancement for UDP traffic. 

In our experiment setup, we have two PCs while each PC has two NIC cards, every NIC card works at 100Mbps. With these four NIC cards, we have different configurations to verify our system. For each configuration, we do side-by-side comparision among regular connection, MPTCP(For TCP), and MPIP connection. We use iperf to transmit traffic for five minutes, and we customized iperf to record real-time throughput of each second.

In Figure~\ref{fig.nonat}, we connect the two machine to the same router which means there are no NAT devices on all the paths.

\begin{figure*}[htb]
\centering{
\subfigure[TCP throughput with psudo TCP connection\label{fig.tcp_usetcp_nonat}]{\includegraphics[width=0.33\linewidth]{fig/tcp_usetcp_nonat.eps}}
\subfigure[TCP throughput with UDP Wrapper\label{fig.tcp_useudp_nonat}]{\includegraphics[width=0.33\linewidth]{fig/tcp_useudp_nonat.eps}}
\subfigure[UDP throughput\label{fig.udp_nonat}]{\includegraphics[width=0.33\linewidth]{fig/udp_nonat.eps}}
}
\caption{Side-by-side comparison without NAT}
\label{fig.nonat}
\end{figure*}

Also, we set up a NAT network with two routers as shown in Figure~\ref{fig.nat}. Because our router only has 100Mbps capacity while each NIC has 100Mbps, this means that our router can't fit all capacity of the NIC cards. To avoid this problem, we limit the bandwidth of our NIC cards to be 5Mbps with WonderShaper. Then the total capacity of our connection is 10Mbps. Figure~\ref{fig.nat} shows the result for this configuration.

\begin{figure*}[htb]
\centering{
\subfigure[TCP throughput with psudo TCP connection\label{fig.tcp_usetcp_nat}]{\includegraphics[width=0.33\linewidth]{fig/tcp_usetcp_nat.eps}}
\subfigure[TCP throughput with UDP Wrapper\label{fig.tcp_useudp_nat}]{\includegraphics[width=0.33\linewidth]{fig/tcp_useudp_nat.eps}}
\subfigure[UDP throughput\label{fig.udp_nat}]{\includegraphics[width=0.33\linewidth]{fig/udp_nat.eps}}
}
\caption{Side-by-side comparison with a simple NAT}
\label{fig.nat}
\end{figure*}


In Figure~\ref{fig.net}, we set up our server in Emulab to verify our system in Internet. Because the server in Emulab only has one public IP address which means our connection only has two paths. Also, Emulab doesn't support iperf with UDP traffic, so we only test TCP scenario.

\begin{figure*}[htb]
\centering{
\subfigure[TCP throughput with psudo TCP connection\label{fig.tcp_usetcp_net}]{\includegraphics[width=0.49\linewidth]{fig/tcp_usetcp_net.eps}}
\subfigure[TCP throughput with UDP Wrapper\label{fig.tcp_useudp_net}]{\includegraphics[width=0.49\linewidth]{fig/tcp_useudp_net.eps}}
}
\caption{Side-by-side comparison in Internet}
\label{fig.net}
\end{figure*}


%\subsection{Skype voice call improvement}
%\label{sec:skype}
%
%\subsection{Datacenter use case}
%\label{sec:datacenter}
%
%\subsection{Seamless handover with dynamic networks}
%\label{sec:handover}
