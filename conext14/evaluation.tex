%!TEX root =conext14.tex
\section{Performance Evaluation}
\label{sec:evaluation}
To evaluate the performance of our proposed system, we implemented our multipath IP in Ubuntu under Linux kernel $3.12.1$. Two desktops are installed with the MPIP enabled Ubuntu system. At each desktop, two NICs working at $25$Mbits/sec are installed which means that there are totally $4$ paths and the throughput upper bound is $50$Mbits/sec between the two nodes. We modify an open source software \bf{Simple Traffic Monitor}\cite{simon01} to record the real time traffic that goes through each NIC.

%For mobile devices, we implement this feature in Google Nexus $4$ with Android $4.01$.

\subsection{TCP/UDP throughput enhancement}
\label{sec:tcp}

In this section, we try to verify that our system can achieve high throughput in both TCP and UDP scenarios. As a typical implementation of multipath, MPTCP creates the highest TCP throughput record between two nodes in \cite{record}. In that demonstration, they used $6$ $1$Gbps NIC interfaces at each node, and connect them directly without only middle boxes, and they limited the number of paths to $6$ by setting up IP TABLES in Ubuntu. Also, they did bunch of TCP parameter optimization to squeeze out all possible throughout. They achieved a breathtaking $51$Gbps throughput in that demonstration. 

For TCP, under the same network configuration, we do a side-by-side comparison among regular TCP, MPTCP and MPIP. For each case, we repeatedly download a large file for $100$ times with wget. Figure~\ref{fig.tcp} shows the comparison among these three scenarios.

\begin{figure}
\centering
\includegraphics[width=0.8\linewidth]{fig/tcp.eps}
\caption{Side-by-side Throughput Comparison for TCP}
\label{fig.tcp}
\end{figure}


Besides TCP, another major category of traffic on the Internet is UDP. To our knowledge, MPIP is the first implementation on end nodes that supports multipath transmission for UDP traffic. We use iperf as the test tool for UDP. At the client, we set the bandwidth to be $200Mbps$ which is larger than the capacity of our test platform. Then iperf will hit the upper bound of the network's capacity. We also repeat the experiment for $100$ times for each scenario. Figure~\ref{fig.udp} shows the comparison among regular UDP and UDP with MPIP enabled.

\begin{figure}
\centering
\includegraphics[width=0.8\linewidth]{fig/udp.eps}
\caption{Side-by-side Throughput Comparison for UDP}
\label{fig.udp}
\end{figure}


\subsection{Skype voice call improvement}
\label{sec:skype}

\subsection{Datacenter use case}
\label{sec:datacenter}

\subsection{Seamless handover with dynamic networks}
\label{sec:handover}
